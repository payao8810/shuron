% vim: set tabstop=4 :
%**********************************************************
\chapter{おわりに}
\label{sec:discuss}
%**********************************************************
本論文では,これまで述べてきた提案手法とその評価結果をまとめ,本研究全体の総括を行う.
本研究では,SFMを用いた人流シミュレーションを高速化するために,
エージェント間距離の計算回数削減手法と,
進行方向の計算回数削減手法を提案し,その有効性を評価した.

第\ref{sec:reduce_distance}章「エージェント間距離の計算回数削減手法」では,
視野パラメータを用いたSFMの人流シミュレーションにおいて,周囲のエージェントが
視野範囲内に存在するかの判定に必要なエージェント間距離の計算回数を削減する手法を提案した.
SFMを用いた人流シミュレーションは,一般的にセル分割法を用いることで,
エージェント間距離の計算回数の削減が行われている.
セル分割法は,影響範囲が円形であることを前提としており,視野を用いたSFMのように影響範囲が
扇形の場合を想定していない.
このため,視野を用いたSFMに一般的なSFMのエージェント間距離の計算回数削減手法を用いると,
影響範囲を円形に絞り込みをした上で,視野形状に合わせた扇形に絞り込みを行う操作が必要となる.
本提案手法では,視野形状である扇形に近似した領域を設定し,
近似領域外のエージェントに対するエージェント間距離の計算回数を削減することで,
視野を用いたSFMを高速化する.
評価の結果,提案手法は,エージェントが交差するように移動する問題において,
十分に許容可能な誤差の範囲で約1.25倍の高速化率が得られることを確認した.
%他にもあればここに追記予定


第\ref{sec:reduce_travel_direction}章「格子分割を用いた進行方向の計算回数削減手法」では,
SFMを用いた人流シミュレーションにおいて,
進行方向をあらかじめ計算することで,解析中の進行方向の計算回数を削減する手法を提案した.
本手法は,目的地や障害物の座標が解析中に変化しない特徴を利用し,解析領域を格子状に分割した
領域ごとに目的地に向かうベクトルと障害物を避ける力を計算し,解析中にその値を用いて
解析する
評価の結果,提案手法は,通路を再現した配置を移動する問題において,
エージェントの進行方向の計算回数を最大○○\%削減し,シミュレーション実行時間を
最大○○倍高速に解析できることを確認した.
また,提案手法は,避難シミュレーションにおいても,大学の施設を再現した配置から避難する様子を
再現する解析で十分に許容できる誤差の範囲で最大○○倍高速に解析できることを確認した.

以上の結果より,SFMを用いた人流シミュレーションにおいて,ふたつの手法で計算回数を削減し,
高速化を確認した.第\ref{sec:reduce_distance}章で提案した手法は,視野パラメータを用いた
SFM

\textbf{続く}


%***** END ************************************************


%------------------------------------------------------------------
% 注意点
%------------------------------------------------------------------
% 「おわりに」は,ここまでの文章をすべて読んできた人向けの文章です.
% このため,細かな用語の説明は必要ありません.
% 論文の論理構造が分かるような文章を1\UTF{FF5E}2ページ目安で書いてください.
% 
% 構成例を以下に示します.
% [1段落目]
% 従来手法の問題点とそれに対する提案手法
% [2段落目]
% 提案手法の手順
% [3]段落目以降
% 評価結果とそこから導き出される結論
%
%------------------------------------------------------------------
