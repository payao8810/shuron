%%%%%%%%%%%%%%%%%%%%%%%%%%%%%%%%%%%%%%%%%%%%%%%%%%%%%%%%%%%%%%%%%%
% 前研スタイルファイル 設定部分
%
%    27行目までの項目を用いて表紙を生成しています.
%    学科のabstractスタイルが変更になった場合は,
%    27行目までの項目がすべてそろうようにスタイルファイルに
%    付け足してください
%
%%%%%%%%%%%%%%%%%%%%%%%%%%%%%%%%%%%%%%%%%%%%%%%%%%%%%%%%%%%%%%%%%%

%改行位置はそれぞれ自分で調節する
\表紙題目{
進行方向の計算回数削減による\\SFMを用いた人流シミュレーション\\の高速化
}

%名前の間は半角スペース
\和文氏名{片寄 颯人}
\英文氏名{KATAYOSE Hayato}

\学生番号{2281011}

\令和年度{5}
\西暦年度{2023}

\提出日{2023年12月24日} % 表紙だけじゃなくて謝辞でも使っているので注意


\和文題目{
進行方向の計算回数削減による\\SFMを用いた人流シミュレーションの高速化
}

%%%%%[ここから下は修士のみ記入]%%%%%%%%%%%%%%%%%%%%%%%%%
%
%   アブストラクト用の設定
%       表紙と改行位置が異なる場合があるので注意!

%\修論false        %この行を消去(コメントアウトでもOK)すること

\表紙西暦{2022}
\英文題目{Accelerate matrix sum-of-products operations by\\ improving execution efficiency using \\stream processing in GPU kernels}

\和文キーワード{GPU,カーネル,ストリーム,Tesla V100,Tensorコア}
\英文キーワード{GPU,Kernel,Stream,Tesla V100,TensorCore}

\和文論文要旨={

本論文では,CUDAを用いた行列積和演算を高速化するために,CUDA機能の一つである
ストリームを用いてカーネルを複数発行し,実行スレッドブロック,ワープ数を増加することでGPUの計算実行効
率を向上する手法を提案する.

CUDAを用いた行列積和演算は,多数のスレッドを起動し,各要素の計算を並列実行する.
CUDAを用いた数値計算において,GPUの性能を最大限引き出すには,より多数のスレッドを起動し,
スレッドのまとまりであるワープやスレッドブロックをSMを占有された状態になるように確保する必要がある.

GPUアーキテクチャに最適化した手法として,cuBLASやCUTLASSといったGPU向け数値計算ライブラリがある.
 これらのライブラリでは,計算カーネルの実行性能を高めるために,レジスタやシェアードメモリといったリソースを
 最大限使用する行列積和ルーチンの実装が行われており,問題サイズが大きくなるほど使用するリソースが増加し,
リソースの制限により生成されるスレッドブロック数や,発行ワープ数が減少し,GPU実行効率が低下する.

そこで提案手法は,GPUのSM実行効率を向上するために,行列積和処理を分割し,実行カーネル数を増やし並列実行することで,
行列積和計算時間を短縮する.行列積和処理カーネルを,リソースを最大限利用して計算を行うカーネルと,使用リソースを最小限に抑えたカーネルに分割する.
各カーネルをストリーム処理し,SMにより多くのスレッドブロックを割り当てることによってSM実行効率を向上する.

 最後に,提案手法をNVIDIA Tesla V100で評価した.行列積和演算を対象とした評価の結果,
 ストリーム処理を使い計算カーネルを並列実行する手法は,Tensorコアのみを用いた手法と比べ,行列サイズ32000で最大約1.09倍高速化することが確認できた.
}% 600字程度


\英文論文要旨={

Write summary here.

}% about 200 words
