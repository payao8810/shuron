% vim: set tabstop=4 :
%**********************************************************
\chapter{プログラムの説明}
\label{sec:appendix}
%**********************************************************
付録には,添付するソースコードの説明を書いてください.
データ構造や主要な変数の説明は本文中で述べてあると思います.
本文で述べたことを一覧形式でまとめる分には構いませんが,
まったく同じことを書くのはよくありません.
このため,本文中では書けない実装の話
(コンパイル方法や, 測定条件の変更方法,入出力フォーマットなど)
を中心に書きましょう.

また,付録のページは,
本文中で邪魔になった定義とか証明とかの避難場所としても利用可能です.

\section{節番号のテスト}
\subsection{項番号のテスト}
付録では,こんな風に章番号が表示されます.
付録A, 付録Bというように,付録のchapterにも章番号をつけたい場合は,
main.tex66行目の$\setminus$appendixを$\setminus$appendixesに変更してください.


%***** END ************************************************