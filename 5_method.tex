% vim: set tabstop=4 :
%**********************************************************
%\chapter{提案手法にあたる章}
%\chapter{格子分割を用いた進行方向計算の削減手法}
\chapter{格子分割による進行方向ベクトル計算の削減手法}
\label{sec:method}
%**********************************************************
\section{本章の概要}
SFMの運動方程式は,エージェントの位置が変わるたびに,
目的地までのベクトルを表す$e$や周囲のエージェントを避ける力$f_{ij}$,
障害物を避ける力$f_{iW}$の再計算が必要となる.
周囲のエージェントを避ける力$f_{ij}$の計算は,時間ステップごとにすべての
エージェントの位置が変化するため,解析中のみに計算が可能である.
一方,ベクトル$e$や障害物を避ける力$f_{iW}$は,エージェントの位置に
応じて決定し,壁や机などの障害物の座標が固定であることから,解析前に計算が可能である.

そこで,本論文では,格子状に分割した領域ごとにベクトル$e$と障害物を避ける力$f_{iW}$を
あらかじめ計算することで,解析中の進行方向の再計算を削減する.

\section{格子分割を用いた進行方向計算手法}
本手法は,解析領域を格子状に分割し,格子ごとに進行方向ベクトル$e$や
障害物を避ける力$f_{iW}$をあらかじめ計算する.
進行方向ベクトル$e$は,式\eqref{eq:sfm_siki1}に示すSFMの運動方程式から
$v_i^0(t)$が係数である.一方で,障害物から受ける力$f_{iW}$は,
式\eqref{eq:sfm_siki1}に示すように,式の第3項である.
このため,進行方向ベクトル$e$と障害物を避ける力$f_{iW}$は,
別々の配列に格納する必要がある.

\figref{fig:5_teian_flow1}に格子分割を用いた
進行方向ベクトル計算の削減手法のフローチャートを示す.
\figref{fig:5_teian_flow1}中の前処理は,提案手法に必要である格子ごとの進行方向や
壁を避ける力を算出する.
\figref{fig:5_teian_flow1}中の運動方程式を用いた計算は,
エージェントの進行方向を計算する処理であり,
エージェントの進行方向ベクトル$e$や
障害物を避ける力$f_{iW}$を前処理で算出した値を参照する.

\figtb{提案手法の解析全体のフローチャート}{}{5}{5_teian_flow1.eps}{5_teian_flow1}

%\subsection{格子分割を用いた進行方向ベクトル$e$の計算方法}
\subsection{格子ごとの進行方向ベクトル$e$の計算方法}
進行方向ベクトル$e$は,目的地に進むベクトルであり,エージェントの座標から
目的地の座標までのベクトルである.
このため,ベクトル$e$は解析領域を格子状に分割した領域ごとの中心座標から
目的地の座標までのベクトルを計算することで,解析前に計算が可能となる.
\figref{fig:grid_ex1}に格子分割した進行方向の例を示す.
%
\figtb{提案する格子分割の例}{}{9}{ex1.eps}{grid_ex1}
%
\figref{fig:grid_ex1}中の○はエージェントであり,エージェントA,Bが1つの目的地に向かう
様子である.
図中のエージェントBは,エージェントAの位置に移動したとき,エージェントAと同じ進行方向に
なる.
このため,解析領域を分割し,格子ごとに進行方向を算出することで,解析中の
進行方向の再計算が削減できる.	
目的地が含まれる格子は,エージェントが正しく目的地に進むようにするために,
0ベクトルを格納する.
0ベクトルが格納された格子中に存在するエージェントは,
改めて個別に進行方向ベクトル$e$を計算する.
格子ごとの進行方向は,格子の中心座標から目的地までのベクトル$e$を求め,配列に格納する.
進行方向は,経由地ごとに異なるため,経由地ごとに違う配列に格納する必要がある.
経由地ごとに別の配列に進行方向を格納するため,進行方向を格納する配列の要素数は
式\eqref{eq:route_youso_size}に進行方向を格納する要素数を示す.
%
\begin{eqnarray}
 \mbox{進行方向を格納する格子の要素数[個]} =  \Big( \frac{\mbox{解析領域[m]}}{\mbox{格子サイズ[m]}} \Big) ^ 2 \times  \mbox{経由地数[個]}
 \label{eq:route_youso_size}
\end{eqnarray}
%
式\eqref{eq:route_youso_size}のように,進行方向を格納する配列は,
解析領域の大きさや格子サイズの小ささ,経由地数の大きさに応じて
要素数が増加する.
また,進行方向を格納する配列の前処理にかかる時間は,
配列の要素数に応じて格子ごとの進行方向計算回数が増加するため,
配列の大きさに応じて増加する.

%%工事中

%\subsection{格子分割を用いた障害物を避ける力$F_{iW}$の計算方法}
\subsection{格子ごとの障害物を避ける力$F_{iW}$の計算方法}
格子ごとの進行方向は,格子の中心座標から目的地までのベクトル$e$を求め,配列に格納する.同様に,
壁から受ける力$f_{iW}$は,解析中に机や壁が動かない特徴を利用し,格子ごとの中心座標を用いて,
$f_{iW}$をあらかじめ算出する.
目的地や障害物が存在する格子では,正しく目的地に進むように,0ベクトルを格納する.0ベクトルが格納されている格子中に存在する
エージェントは,エージェントごとに本来のSFMと同様に進行方向を計算する.
目的地の他に経由地が複数存在する場合は,\figref{fig:ex2}に示すように,経由地ごとに進行方向を
設定する.
\figref{fig:ex2}の例では,目的地に向かっているエージェントは,目的地の進行方向が格納されている配列を
参照し,経由地に向かっているエージェントは,経由地の進行方向が格納されている配列を参照する.

\figref{fig:5_teian_flow1}に提案手法の解析全体のフローチャートを示す.
\figref{fig:5_teian_flow2}に提案手法の前処理のフローチャートを示す.
\figref{fig:5_teian_flow2}に示すように,提案手法は,進行方向のベクトル$e$を
格納する配列と壁を避ける力$F_W$を格納する配列が必要となる.
各配列の計算は,\figref{fig:5_teian_flow2}のように格子ごとにあらかじめ計算する.
進行方向のベクトル$e$を格納する配列中の要素は,\figref{fig:ex2}のように経由地ごとに
違うため,それぞれの経由地で計算する必要となる.
このため,進行方向ベクトル$e$を格納する配列の前処理は,経由地数の多さや
格子サイズの細かさに応じて計算回数が増加する.
\figref{fig:5_teian_flow2}中の壁を避ける力$f_{iw}$を格納する配列は,
経由地が変わっても変化しないため,解析領域全体で一つとなる.
このため,壁を避ける力$f_{iw}$を格納する配列の前処理は,格子サイズの細かさに応じて
計算回数が増加する.

\figtb{経由地がある場合の進行方向の例}{An example of proposed method with waypoints.}{7}{ex2.eps}{ex2}
\figtb{提案手法の前処理のフローチャート}{}{5}{5_teian_flow2.eps}{5_teian_flow2}
\figtb{提案手法の運動方程式計算のフローチャート}{}{9}{5_teian_flow3.eps}{5_teian_flow2}


\begin{eqnarray}
 \mbox{障害物を避ける力を格納する格子の要素数[個]} =  \Big( \frac{\mbox{解析領域[m]}}{\mbox{格子サイズ[m]}} \Big) ^ 2
 \label{eq:fiw_youso_size}
\end{eqnarray}

\subsection{格子分割を用いたSFM}
工事中.

\section{評価}
工事中.

\section{本章のまとめ}
工事中.

%***** END ************************************************
