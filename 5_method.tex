% vim: set tabstop=4 :
%**********************************************************
%\chapter{提案手法にあたる章}
%\chapter{格子分割を用いた進行方向計算の削減手法}
\chapter{格子分割による進行方向ベクトル計算の削減手法}
\label{sec:method}
%**********************************************************
SFMの運動方程式は,エージェントの位置が変わるたびに,
目的地までのベクトルを表す$e$や周囲のエージェントを避ける力$f_{ij}$,
障害物を避ける力$f_{iW}$の再計算が必要となる.
周囲のエージェントを避ける力$f_{ij}$の計算は,時間ステップごとにすべての
エージェントの位置が変化するため,解析中のみに計算が可能である.
一方,ベクトル$e$や障害物を避ける力$f_{iW}$は,エージェントの位置に
応じて決定し,壁や机などの障害物の座標が固定であることから,解析前に計算が可能である.
そこで,本論文では,格子状に分割した領域ごとにベクトル$e$と障害物を避ける力$f_{iW}$を
あらかじめ計算することで,解析中の進行方向の再計算を削減する.

\section{進行方向計算の前処理}

\section{壁を避ける力の前処理}

\figref{fig:ex1}に格子分割した進行方向の例を示す.
\figref{fig:ex1}中の○はエージェントであり,エージェントA,Bが1つの目的地に向かう
様子である.
図中のエージェントBは,エージェントAの位置に移動したとき,エージェントAと同じ進行方向に
なる.
このため,解析領域を分割し,格子ごとに進行方向を算出することで,解析中の
進行方向の再計算が削減できる.	
格子ごとの進行方向は,格子の中心座標から目的地までのベクトル$e$を求め,配列に格納する.同様に,
壁から受ける力$f_{iW}$は,解析中に机や壁が動かない特徴を利用し,格子ごとの中心座標を用いて,
$f_{iW}$をあらかじめ算出する.
目的地や障害物が存在する格子では,正しく目的地に進むように,0ベクトルを格納する.0ベクトルが格納されている格子中に存在する
エージェントは,エージェントごとに本来のSFMと同様に進行方向を計算する.
目的地の他に経由地が複数存在する場合は,\figref{fig:ex2}に示すように,経由地ごとに進行方向を
設定する.
\figref{fig:ex2}の例では,目的地に向かっているエージェントは,目的地の進行方向が格納されている配列を
参照し,経由地に向かっているエージェントは,経由地の進行方向が格納されている配列を参照する.

\figref{fig:5_teian_flow1}に提案手法の解析全体のフローチャートを示す.
\figref{fig:5_teian_flow1}中の前処理は,提案手法に必要である格子ごとの進行方向や
壁を避ける力を算出する.
\figref{fig:5_teian_flow2}に提案手法の前処理のフローチャートを示す.
\figref{fig:5_teian_flow2}に示すように,提案手法は,進行方向のベクトル$e$を
格納する配列と壁を避ける力$F_W$を格納する配列が必要となる.
各配列の計算は,\figref{fig:5_teian_flow2}のように格子ごとにあらかじめ計算する.
進行方向のベクトル$e$を格納する配列中の要素は,\figref{fig:ex2}のように経由地ごとに
違うため,それぞれの経由地で計算する必要となる.
このため,進行方向ベクトル$e$を格納する配列の前処理は,経由地数の多さや
格子サイズの細かさに応じて計算回数が増加する.
\figref{fig:5_teian_flow2}中の壁を避ける力$f_{iw}$を格納する配列は,
経由地が変わっても変化しないため,解析領域全体で一つとなる.
このため,壁を避ける力$f_{iw}$を格納する配列の前処理は,格子サイズの細かさに応じて
計算回数が増加する.

\figtb{提案する格子分割の例}{An example of proposed grid division.}{7}{ex1.eps}{ex1}
\figtb{経由地がある場合の進行方向の例}{An example of proposed method with waypoints.}{7}{ex2.eps}{ex2}
\figtb{提案手法の解析全体のフローチャート}{}{5}{5_teian_flow1.eps}{5_teian_flow1}
\figtb{提案手法の前処理のフローチャート}{}{5}{5_teian_flow2.eps}{5_teian_flow2}
\figtb{提案手法の運動方程式計算のフローチャート}{}{9}{5_teian_flow3.eps}{5_teian_flow2}

%***** END ************************************************
