% vim: set tabstop=4 :
%**********************************************************
%\chapter{提案手法にあたる章}
%\chapter{格子分割を用いた進行方向計算の削減手法}
\chapter{エージェント間距離の削減手法}
\label{sec:method}
%**********************************************************
SFMの運動方程式では,エージェント間距離が大きいほどエージェント間に働く相互作用力が小さくなる.
このため,SFMでは,影響半径$R_{c}$を用いてエージェント間距離の計算回数を削減するのが一般的である.
\figref{fig:sougo_hani}に,エージェント4の影響半径の例を示す.
\figref{fig:sougo_hani}中の〇はエージェントであり,
オレンジ色の点線で囲まれた領域がエージェント4の影響範囲である.
影響半径$R_{c}$は影響範囲の半径であり,
影響範囲外から受ける力を0とすることで,
運動方程式を計算する際に演算を省略することができる.
一方,視野を用いたSFMでは,視野範囲外の相互作用力を0とする.
\figref{fig:sougo_hani}の例で,エージェント4が図中矢印の方向に移動する際には,
エージェント4の視野は,図中の緑色の領域のように設定され,
視野$\subseteq$影響範囲のような関係となる.

SFMで用いられるエージェント間距離の計算回数削減手法は,
影響範囲が円形であることを前提としており,
影響範囲が視野のように扇形の場合を想定していない.
例えば,\figref{fig:sougo_hani}のエージェント配置では,
半径$R_{c}$内の白い領域に存在する5つのエージェントが視野外にあるにも関わらず,
これらのエージェントに対するエージェント間距離の計算が必要となる.
人の視野角$\theta$は$\pi$以下であるため,
視野を用いたSFMに一般的なSFMのエージェント間距離の計算回数削減手法を用いると,
影響範囲を円形に絞り込みをした上で,%視野形状に合わせた
扇形に絞り込みを行う操作が必要となる.
このため,視野範囲外となる半径$R_{c}$内のエージェントが多く存在するほど,
相互作用力を0として計算するエージェントに対するエージェント間距離の計算回数が増える.
そこで,本論文では,
視野形状である扇形に近似した領域を設定し,
近似領域外のエージェントに対するエージェント間距離の計算回数を削減することで,
視野を用いたSFMを高速化する.

提案手法では,近似領域の算出に必要な計算時間を最小限に抑えるために,
長方形で視野範囲を近似する.
これに合わせて,
視野範囲はエージェントの進行方向前方に存在するため,
エージェントの進行方向を長方形の辺数に合わせて上下左右の4パターンに分類し,
エージェントの進行方向に応じた近似領域を設定する.
エージェントの進行方向の分類方法によりシミュレーション時間が変化する可能性があるため,
本論文では,既存手法であるセル分割法を含む6パターン実装し,その有効性を評価する.
%パターン2~6は,パターン1のセル分割法で用いたセルを利用しており,
%各手法の影響範囲の設定条件は\tabref{tb:hantei_jouken}の通りである.
提案手法であるパターン2~6は,いずれの手法もパターン1のセル分割法の考え方を基にしており,
時間ステップごとに影響範囲の近似領域を設定した上で,
近似領域内で影響範囲内となるエージェントを相互作用力を計算する対象として設定する.
\tabref{tb:hantei_jouken}に,エージェントの進行方向を分類する条件を示す.
ただし,\tabref{tb:hantei_jouken}で条件に数式が2つ記述されているものは,
両条件を満たす必要があることを表す.
また,
\figref{fig:90do_hamideru}に,\figref{fig:sougo_hani}のエージェント4に対して
\tabref{tb:hantei_jouken}の各実装パターンが設定する視野範囲の近似領域を示す.

\figtb{影響範囲の例}{An example of the scope of influence.}{3.7}{20230614_hanni_ex.eps}{sougo_hani}

\figtb{エージェント4の実装パターンごとの近似領域の例}{The approximation region for each pattern of agent 4.}{8}{20231007_hanni.eps}{90do_hamideru}

\figtb{進行方向ごとの近似領域}{The approximation regions for each direction.}{7}{20220225_sentaku.eps}{sentaku}

\begin{table*}[tb]
\begin{center}
\caption{進行方向を分類する条件}
\label{tb:hantei_jouken}
\begin{tabular}{c|c|c|c|c}
\hline \hline
			& 右 & 左 & 上 & 下 \\ \hline
パターン2   & $\frac{1}{\sqrt{2}} < e_x \leq 1  $
		    & $ -1 \leq e_x < \frac{-1}{\sqrt{2}}$ 
		    & $ \frac{-1}{\sqrt{2}} < e_x < \frac{1}{\sqrt{2}} $ 
		    & $ \frac{-1}{2} < e_x < \frac{1}{2} $ \\
パターン3   & $\frac{-1}{2} < e_y < \frac{1}{2} $ 
		    & $\frac{-1}{2} < e_y < \frac{1}{2} $
            & $ \frac{1}{\sqrt{2}} < e_y \leq 1$
		    & $ -1 \leq e_y < \frac{-1}{\sqrt{2}} $ \\
\hline
\multirow{2}{*}{パターン4}   
			& $R_x \geq A_x$ & $R_x < A_x$ & $R_y \geq A_y$ & $R_y < A_y $ \\
	        &  $L_x \geq A_x$ & $L_x < A_x$ & $L_y \geq A_y$ & $L_y < A_y$ \\
\hline
\multirow{2}{*}{パターン5}   
 			& $R_x \geq x_1$ & $R_x < x_2$ & $R_y \geq y_1$ & $R_y < y_2 $ \\
			& $L_x \geq x_1$ & $L_x < x_2$ & $L_y \geq y_1$ & $L_y < y_2 $ \\
\hline
パターン6   & $ \cos(\frac{1}{2}\theta_{view}) \leq  e_y $ 
			& $ e_y \leq -\cos(\frac{1}{2}\theta_{view})$ 
			& $ \sin(\frac{1}{2}(\pi - \theta_{view})) \leq e_x $ 
			& $ e_x \leq \sin(\frac{1}{2}(\pi - \theta_{view}))  $ \\
\hline
\end{tabular}
\end{center}
\end{table*}

\section{パターン1(セル分割法)}
パターン1は,一般的なセル分割法\cite{cell1},\cite{cell2}によるエージェント間距離の計算回数削減を行う.
\figref{fig:90do_hamideru}では黄色の領域がセル分割法による近似領域を表す.
セル分割法は,\figref{fig:90do_hamideru}のように解析領域を格子状のセルに分割し,
影響半径$R_{c}$の円内に重なるセルを近似領域とする手法である.
本手法は,分割するセルのサイズを影響半径と同じ距離に設定することで,
近似領域をエージェントの近傍9セルに限定することができる.
本例では,エージェント4の近傍9セルは黄色のセルであり,
それ以外のセルに属するエージェント3,5,9に対するエージェント間距離の計算を削減できる.
なお,セル分割法は相互作用力を計算する範囲を円形で想定した手法であるため,
近似領域が正方形となることから,パターン1の実装では進行方向の推測は行わない.
また,本論文では,セル分割法のなかでもメモリ使用量を抑えることができると知られている
連結リスト法\cite{cell_book1},\cite{cell_renketu}を用いる.

\section{パターン2}
パターン2は,エージェントの進行方向を表す単位ベクトル$e_{i}$を参照することで,
パターン1の近似領域を視野範囲に近づける手法である.
本手法は,\tabref{tb:hantei_jouken}の条件に基づいて
単位ベクトル$e_{i}$が示す進行方向を上下左右を$\frac{1}{2}\pi$の均等な角度で分割し,
セル分割法の近似領域から\figref{fig:sentaku}のように進行方向の反対側にある3セルを除外する.
\figref{fig:90do_hamideru}の例では,エージェントの進行方向は右と判定され,
\figref{fig:sentaku}に示す右方向の青いセルを近似領域に設定する.
このため,白色のセルに存在するエージェントに対する距離計算を削減できる.
パターン2は,進行方向を必ず上下左右のいずれかに設定するため,
常に近似領域が6セルとなり,セル分割法の9セルに比べて近似領域の面積を$\frac{2}{3}$に削減できる.
一方で,\figref{fig:90do_hamideru}のように視野範囲全体が近似領域内に含まれる保障がないため,
本来相互作用力の計算が必要なエージェントに対する計算を削減し,誤差が発生する可能性がある.

\section{パターン3}
パターン3は,パターン2で発生する誤差を防ぐために,
近似領域外に視野範囲が存在しないかを判定する処理を追加し,
セル分割法と同じ精度を保つ手法である\cite{katayose}.
本手法は,パターン2の近似領域を導出したのち,
視野の座標$(L_x,L_y)$,$(R_x,R_y)$が
近似領域内のセルであればパターン2の近似領域を用いて相互作用力を計算し,
そうでなければパターン1の近似領域を用いて相互作用力を計算する.
\figref{fig:90do_hamideru}のエージェントは,
$L_{x}, L_{y}$がパターン2の近似領域の外にあるため,
パターン3ではパターン1と同様に近傍9セルを近似領域とする.
エージェントの座標は時間ステップごとに変化するため,
パターン3を用いると,進行方向が変化しないエージェントに対しても
すべての時間ステップでパターン1よりもエージェント間距離の計算を削減できるとは限らない.
一方で,パターン3は,パターン1のセル分割法と同じ精度のシミュレーション結果を得ることができる.

\section{パターン4}
パターン4は,視野の左右両端の座標$(L_x,L_y)$,$(R_x,R_y)$を用いて近似領域を設定する手法である.
視野範囲は進行方向の前方に存在するため,
視野の左右両端の座標はエージェント座標から見て必ず進行方向側となる.
この特性を利用し,パターン4では,\tabref{tb:hantei_jouken}のように,
エージェント座標$(A_x,A_y)$と視野座標$(R_x,R_y)$,$(L_x,L_y)$の
大小関係を用いて進行方向を判定する.
\figref{fig:90do_hamideru}の例では,$R_x \geq A_x$および$L_x \geq A_x$が成り立つため,
エージェントの進行方向は上と判定される.
進行方向が求まると,近傍9セルから進行方向反対側の3セルを除外した6セルを近似領域とする.
\tabref{tb:hantei_jouken}のいずれの条件にも当てはまらない場合は,
パターン1(セル分割法)と同様に近傍9セルを近似領域とする.

\subsection{パターン5}
パターン5は,セル分割法のセル座標と視野範囲の左右両端の座標$(L_x,L_y)$,$(R_x,R_y)$を
用いて近似領域を設定する手法である.
本手法は,運動方程式を算出するエージェントが所属するセルの
左上座標$(x_1,y_2)$および右下座標$(x_2, y_1)$を用いて,
\figref{fig:sentaku}の4パターンの水色の領域内に視野が収まっているかを判定する.
\figref{fig:90do_hamideru}の例では,
$R_y \geq y_1$および$L_y \geq y_2$が成り立つため,進行方向は上と分類される.
本手法は,エージェントのセル上の位置に応じて進行方向を分類するため,
ベクトル$e_i$の表す進行方向が同じエージェントでも
エージェント座標に応じて異なる方向に分類される可能性がある.
本手法も\tabref{tb:hantei_jouken}のいずれの条件にも当てはまらない場合は,
パターン1(セル分割法)と同様に近傍9セルを近似領域とする.

\section{パターン6}
パターン6は,パターン2で設定した上下左右の角度範囲を,
シミュレーション誤差の出ない範囲で設定する手法である.
パターン6では,
\figref{fig:sentaku}中の水色の領域である近似領域内に視野範囲がすべて収まる進行方向を静的に計算し,
エージェントの進行方向を表すベクトル$e_i$の閾値を定める.
視野角を$\theta_{view}$とおくと,
エージェント座標がどの位置であっても上方向と判定される進行方向の角度は
$\frac{1}{2}(\pi - \theta_{view})$から$\frac{1}{2}(\pi + \theta_{view})$の間であり,
これを用いて進行方向ベクトル$e_{i} = (e_{x}, e_{y})$の範囲が
$\sin{(\frac{1}{2}(\pi - \theta_{view}))} \leq e_{x}$という条件を設定できる.
パターン6は,パターン3~6のように視野座標$(R_x,R_y)$および$(L_x,L_y)$を算出する必要がないため,
他の分類条件よりも高速に進行方向を分類することができる.
一方で,進行方向が特定の方向である場合は,
エージェント座標に関係なくパターン1(セル分割法)と同じ近似領域を設定するため,
エージェント間距離の計算回数は,パターン1に次いで多くなると考えられる.

%***** END ************************************************
