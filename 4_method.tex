% vim: set tabstop=4 :
%**********************************************************
\chapter{提案手法にあたる章}
\chapter{格子分割を用いた進行方向計算の削減手法}
\label{sec:method}
%**********************************************************

\section{ファイル構成}


\tabref{tb:filelist}に,zipファイル中のファイル一覧を示します.
\tabref{tb:filelist}中の記号の意味を以下に示します.
\begin{description}
\item[◎]編集してはいけない大切なファイル
\item[○]全員が編集するファイル
\item[△]状況に合わせて編集
\item[×]使わない
\end{description}

表紙に記述する情報の設定する際は,abstract.texを書き換えてください.
また,表紙フォーマットを変更したい場合は,
penguinB4.styおよびpenguinM2.styの250行目くらいを書き換えてください.



\begin{table}[htb]
\begin{center}
\caption{ファイル一覧}
\label{tb:filelist}
{\tabcolsep=0.3cm
\begin{tabular}{l|l|c|c}
\hline\hline
ファイル名 & 内容 & B4 & M2 \\ \hline
figure/     & 図を入れておくためのディレクトリ  & & \\ \hline
sty/        & スタイルファイルが多いのでまとめた & & \\ \hline
abs\_ sample/ & & & \\  \hline
1\_ intro.tex & main.texが呼び出すファイル& & \\  
2\_ background.tex & 研究内容に合わせて章構成を決めてください& & \\ 
3\_ survey.tex & &○&○\\ 
4\_ method.tex & & & \\ 
5\_ result.tex & & & \\ 
6\_ discuss.tex & & & \\ \hline
astract.tex & 表紙情報と修論アブストラクト&○&○\\ \hline
appendix.tex & 付録を書く&○&○\\ \hline
bibfile.bib & bibファイル使用時は,ここに記述する &△&△\\ \hline
bibfile.tex & bibファイル未使用時は,参考文献をここに書く &△&△\\ \hline
cover.tex   & 表紙を作るためのファイル &◎&◎\\ \hline
ils.mf & 修論テンプレートに入っていたファイル(未変更)  &×&◎\\ \hline
ipsjunsrt.bst & 情処の参考文献スタイルファイル &◎&◎\\ \hline
main.tex  & platexでコンパイルするtexファイル &〇&〇\\ \hline
makefile    & &△&○\\ \hline
penguin.sty &中村さんが作ったスタイルファイル &◎&◎\\ \hline
shuronABS.sty &学科の修論abstractスタイルファイル&◎&◎\\ \hline
thanks.tex    & 謝辞を書くところ &○&○\\ \hline
\end{tabular}}
\end{center}
\end{table}



\section{makeコマンドの使い方}

\tabref{tb:6_arc}に,texファイルをコンパイルするためのコマンドを示します.
生成されるpdfファイルは以下の2種類です.
\begin{description}
\item[main.pdf (main.dvi)] 図書館提出用の修論データ
\item[cover.pdf (cover.dvi) ] 学科提出用ファイルの表紙に張り付けるためのデータ
\end{description}

\begin{table}[htb]
\begin{center}
\caption{makeコマンドの使い方について}
\label{tb:6_arc}
{\tabcolsep=0.3cm
\begin{tabular}{l|l|l}
\hline\hline
%\multicolumn{2}{|c}{コマンド}
コマンド  & 効果 & 生成ファイル \\
\hline
make       & 論文データを2回コンパイル & main.pdf   \\
                   & 印刷用pdf作成                  & cover.pdf  \\                  \hline
make cover & 学科提出用の表紙のみ作成  & cover.pdf  \\                      \hline
make dvi   & 論文データを2回コンパイル & main.dvi   \\                      \hline
make clean & dviファイルを作るために作成したファイルを削除& \\       \hline 
\end{tabular}}
\end{center}
\end{table}




%***** END ************************************************
