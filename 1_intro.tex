% vim: set tabstop=4 :
%**********************************************************
\chapter{はじめに}
\label{sec:intro}
%**********************************************************
%\subsection{人流シミュレーションの背景と需要}
人が多く集まるイベントなどの場所では,想定よりも多くの人が集まることで,
群集事故が発生する恐れがある.
群集事故は,


商業施設やイベント会場などの人が多く集まる場所では,
災害時の逃げ遅れの観点から
人の滞留の対策が重要であり\cite{taisaku1},\cite{taisaku2},
人の滞留や避難時間の予測に人流シミュレーションが用いられている
\cite{sim_jirei1},\cite{sim_jirei2},\cite{sim_jirei3},\cite{sim_jirei8}.
人流シミュレーションは,コンピュータ上で人を運動方程式に基づくエージェントとして
解析する手法である.
人流シミュレーションのなかでも,歩行者の動きの再現には,視野やグループ特性などの
パラメータを追加できるSocial Force Model(SFM)が広く用いられている
\cite{helbing_sfm},\cite{sfm_ntt},\cite{sfm_para1},\cite{intro_gunshu}.
%\cite{siya_ex5},\cite{siya_ex4},\cite{siya_ex6}.

%\subsection{ソーシャルフォースモデル}
SFMは,社会心理学的な要素と物理学的な要素で成り立つ運動方程式をエージェントごとに計算する
ことで,人流の動きを再現する手法である.
SFMの運動方程式は,目的地に向かう力,周囲のエージェントを避ける力,障害物を避ける力の
合力を算出し,エージェントの速度や進行方向を計算する.
SFMの運動方程式の計算は,時間ステップごとに全てのエージェントに対して計算するため,
エージェント数の増加するほど,解析時間が膨大になることから高速化が求められている.

%\subsection{ソーシャルフォースモデルの高速化技法}
SFMでは,解析時間の高速化をするために,
モデルの1次元化や
エージェント間距離の計算回数の削減が行われている.
%
%\subsubsection{モデルの単純化}
SFMは,エージェントの動きを1次元に簡略化することで,
計算負荷を削減できる
\cite{1ji_sfm1},\cite{1ji_sfm2}.
SFMの1次元化は,避難人数や避難時間などの解析に対して許容できる
範囲の誤差で高速に解析ができるが,滞留の様子や人の密度などの解析ができないことが
報告されている\cite{1ji_sfm1}.
%
%\subsubsection{エージェント間距離の計算回数の削減}
エージェントを避ける力の計算には,エージェント間の距離が必要である.
エージェント間距離の計算回数の削減には,影響半径の設定や,
セル分割法が広く用いられる
\cite{seru_sfm1},\cite{seru_sfm2},\cite{katayose}.
影響半径の設定は,周囲のエージェントを避ける力や障害物を避ける力の
影響力が遠くなるほど0に近づく特性を利用し,影響半径外から受ける力を0に
近似することで,エージェント間距離の計算回数を減らす手法である.
セル分割法は,解析領域を格子状のセルに分割し,
周囲のエージェントに対する影響範囲内外の判定をセル単位で実行する方法である.
影響範囲内外の判定には,エージェント間距離の計算が必要となるため,
セル単位で判定することで,エージェント間距離の計算回数を削減する.

%\subsection{提案手法}
避難時を再現する人流シミュレーションは,机や壁などの障害物が多いため,
障害物を避ける力の計算回数が多い傾向がある.
机や壁などの固定された物である障害物や目的地は,解析中に座標が変化しないという特徴があり,
目的地まで向かう力を計算するために必要なエージェントから目的地までのベクトルは,
エージェントの座標に応じて決定するという特徴がある.
そこで,本論文では,解析前に目的地までの方向と障害物を避ける力の計算をあらかじめ計算し,
メモリに格納することで,
解析中の障害物を避ける力の計算と目的地までのベクトルの計算回数を削減する手法を提案する.
提案手法は,障害物が固定である特徴と目的地までのベクトルがエージェントの座標に応じて決まる特徴に
着目し,解析領域を格子状に分割した領域ごとに進行方向をあらかじめ計算する.
%}}}





以下の章では,まず,ページフォーマットを示すために,
第\ref{sec:background}章で「あああああ」を述べる.
次に,第\ref{sec:survey}章で,本スタイルファイルで定義したコマンドについて述べる.
最後に,題\ref{sec:discuss}章でまとめる.


%***** END ************************************************

%------------------------------------------------------------------
% 注意点
%------------------------------------------------------------------
% 「はじめに」は,文章の導入部です.
% 論理構成を意識し,3年生にも分かるような説明を心がけましょう.
% 
% 研究背景について述べる必要があるため,サーベイの内容を多く書くことになります.
% 参考文献を多く挙げるようにしましょう.
% 
%
% ※当研究室では2ページ以上書かかないと前川先生にOKがもらえません. 
% 
% 
% 
% 一般的には以下のような構成になると思います.
% [1段落目]
% 背景&需要について書きます.
% この研究によって誰が喜ぶのかが分かるように書きましょう.
% [2段落目]
% 従来手法とその問題点について書きます.
% 1段落目の需要があるのに,誰も研究していないなんてことはありえません.
% [3段落目\UTF{FF5E}]
% 従来手法に対する改良手法について書きます.
% 従来手法の問題点が明らかになっているのに,誰も解決しようとしていないなんてありえません.
% 場合によっては手法ごとに段落を分けて書くこともあると思います.
% 研究に直接的に関係のないサーベイ内容も,ここにならがんがん書きましょう.
% [4段落目]
% 従来の改善方法で十分でしょうか?
% 不十分ならどのような手法が必要でしょうか?提案してください.
% [5段落目]
% 提案手法の手順や特徴を述べます.
% [6段落目]
% 「以降の章では\UTF{FF5E}」の文を書きます.
% 箇条書きになりやすいので,注意しましょう.
% 
%------------------------------------------------------------------
