%\newcommand{\tensaku}[1]{#1}
\newcommand{\tensaku}[1]{}

% vim: set tabstop=4 :
%**********************************************************
\chapter{はじめに}
\label{sec:intro}
%**********************************************************
\tensaku{\section{人流シミュレーションの背景と需要}}
駅や商業施設,イベント会場などのように人が多く集まる場所では,
利便性や災害時の逃げ遅れ防止などの安全性の観点から
混雑や滞留の対策が重要であり\cite{taisaku1}\cite{taisaku2},
避難時間の予測や滞留の対策に人流シミュレーションが
用いられている
\cite{sim_jirei1}\cite{sim_jirei2}\cite{sim_jirei3}\cite{sim_jirei8}\cite{sim_jirei7}.
人流シミュレーションは,コンピュータ上で人をエージェントとして,人の動きを再現できる.
人流シミュレーションの歩行者の再現(歩行者モデル)には,
ネットワークモデルや
フロアフィールドモデル\cite{floa_field3}\cite{floa_field1}\cite{floa_field2},
ソーシャルフォースモデル(SFM)\cite{helbing_sfm}などが提案されている.
歩行者モデルのなかでも,SFMは,人が高密度である状態においても高い精度で解析ができることや,
解析用途に応じたパラメータを追加できることから,避難シミュレーションや
交通シミュレーションなどの解析に広く用いられている
\cite{mas_pandemic}\cite{sfm_hinan1}\cite{sfm_hinan2}\cite{sfm_hinan3}\cite{intro_gunshu}.

\tensaku{\section{ソーシャルフォースモデル}}
SFMは,人を運動方程式に基づいて移動するエージェントとして人の動きを再現するモデルである.
SFMの運動方程式は,目的地に向かう力,周囲のエージェントを避ける力,
障害物を避ける力の合力を算出し,エージェントの進行方向や速度を計算する.
SFMは,解析用途に応じてパラメータを追加することで,シミュレーション精度を
高めることができる.
例えば,人の流れを解析する際は,視野を再現するパラメータを追加し,
SFMのシミュレーション精度を向上することが有効であると知られている
\cite{siya_ex2}\cite{siya_ex3}\cite{siya_ex4}\cite{siya_ex5}\cite{siya_ex6}\cite{siya_ex7}.
SFMを用いた人流シミュレーションは,解析人数や経由地や
障害物の数が多くなるほど,
エージェント間の距離や障害物を避ける力の計算回数が多くなり,
エージェントの進行方向の計算に時間がかかるため,
高速化が求められている.

\tensaku{\section{SFMの高速化技法}}
一般的なSFMの高速化技法として,単位時間あたりの計算回数の増加や,
モデルの単純化,
エージェント間距離の計算回数の削減が行われている.
単位時間あたりの計算回数の増加には,並列処理を用いることが一般的である
\cite{seru_sfm1}\cite{seru_sfm2}
\cite{sfm_gpu1}\cite{sfm_gpu2}\cite{sfm_gpu3}\cite{sfm_gpu4}.
SFMは,エージェントごとの運動方程式の計算に並列性があるため,
高い並列性を得ることが可能である\cite{mpi1}\cite{mpi2}.
モデルの単純化は,エージェントの移動を決定するための計算にかかる時間を減らし,
高速化する方法である.
モデルの単純化による高速化手法のひとつにSFMの1次元化手法がある.
1次元化手法は,計算負荷が高いSFMの進行方向の計算をx,yのどちらかに
することで解析時間が削減できる
\cite{1jigen_model}\cite{1jigen_model1}\cite{1jigen_model2},
が出口付近などの滞留の再現度が低いことが報告
されているため,主に避難完了時間や人の流量を解析に用いられる
\cite{1jigen_katuyou}\cite{1jigen_model_ev1}\cite{1jigen_model_ev2}.
エージェント間距離の計算回数の削減には,影響半径の設定や,
セル分割法が提案されている\cite{cell1}\cite{cell2}.
影響半径は,エージェント間の距離が大きくなるほど相互作用力が
小さくなることを利用し,相互作用力が0に近似可能な距離を設定する.
影響半径を設定することで,影響半径外のエージェントに対する
相互作用力が計算不要となる\cite{eikyo_space}.
セル分割法は,解析領域を格子状のセルに分割し,周囲のエージェント
に対する影響範囲内外の判定をセル単位で実行する手法である.
影響範囲内外の判定には,エージェント間距離の計算が必要となるため
複数のエージェントに対する判定処理をまとめて実行することで,
エージェント間距離の計算回数を削減する.

\tensaku{\section{提案手法}}
一般的なSFMの高速化技法は,一般的にエージェントの進行方向計算のなかでも,
目的地に向かう力や周囲のエージェントを避ける力の計算時間削減に
有効である.一方で,建物内の解析は,
壁や机などの障害物が多く存在するため,
障害物を避ける力の計算時間が占める割合が高い傾向がある.
これは,障害物を避ける力の計算時間を削減することで,
さらに解析時間を削減できることを表す.
進行方向計算中の目的地に向かう力や障害物を避ける力は,
解析中に目的地や障害物の座標が変化しないため,
エージェントの座標に応じて決定する特徴がある.
そこで,本論文では,
目的地や障害物の座標が
解析中に変わらない特徴を利用し,
解析領域を格子状に分割した格子領域ごとに進行方向をあらかじめ計算することで,
進行方向計算中の演算回数を削減する手法を提案する.
目的地までのベクトルと障害物を生じる力は,
目的地や障害物の座標に応じて決まるため,
提案手法は,解析前に格子状に分割し,各格子の進行方向をメモリに格納する.
解析中は,メモリに格納した目的地に向かう力や障害物を避ける力を参照することで,
目的地に向かう力や障害物を避ける力の計算回数を削減する.
周囲のエージェントを避ける力は,解析中にそれぞれのエージェントの座標が変化するため,
あらかじめ計算することができない.
提案手法では,
周囲のエージェントを避ける力の
影響範囲内外の判定領域を影響範囲に高い精度で近似することで,
本来不要なエージェント間距離の計算回数を削減する.
影響範囲内外の判定領域を影響範囲に近似する方法は,
エージェントの座標や進行方向などといった複数の選択方法が考えられるため,
本論文では,複数の近似方法を設定し,各手法の有効性を評価する.

\if 0
%\tensaku{\section{エージェント間距離の計算回数を削減する手法}}
エージェント間距離の計算回数を削減する手法は,
視野を用いたSFMの影響範囲に合わせてセル分割法のセルを選択することで,
セル分割法よりもエージェント間距離の計算を削減する.
視野を用いたSFMの影響範囲に合わせたセルの選択は,エージェントの座標や
視野角,進行方向などといった複数の選択方法が考えられる.
このため,本論文では,セルの選択方法を6パターン設定し,
各手法の有効性を評価する.
\fi 

%\tensaku{\section{進行方向の計算回数を削減する手法}}

\tensaku{\section{本論文の構成}}
以下の章では,第2章でSFMを用いた人流シミュレーションの計算方法について
述べる.次に,第3章では,一般的なSFMの高速化技法について述べる.
そして,第4章と第5章で
進行方向計算中の目的地に向かう力,障害物を避ける力の計算回数削減および
エージェント間距離の計算回数削減について述べ,
提案手法が人流シミュレーションに有効であるかを評価するために,
人流シミュレーションの実行時間および計算回数について評価し,最後に第6章で総括する.


\if 0
\clearpage
\section{下書き}
\tensaku{\section{人流シミュレーションの背景と需要}}
滞留などの対策が重要である\cite{taisaku1}\cite{taisaku2}.

感染シミュレーションもあるよ\cite{mas_pandemic}

人が多く集まるイベントなどの場所では,想定よりも多くの人が集まることで,
群集事故が発生する恐れがある.
群集事故は,人が将棋倒しのように倒れる群衆雪崩や〇〇である.
群衆事故を防止するには,急に狭くなる空間を作らないことや階段や出口を
広くするなどが有効である(参考文献).
%https://toyokeizai.net/articles/-/631584?page=3
人の滞留や避難時間の予測に人流シミュレーションが用いられている.
\cite{sim_jirei1}\cite{sim_jirei2}\cite{sim_jirei3}\cite{sim_jirei8}\cite{sim_jirei7}.

人流シミュレーションは,コンピュータ上で人を運動方程式に基づくエージェントとして
解析する手法である.
人流シミュレーションのなかでも歩行者の動きの再現には,
ネットワークモデルやフロアフィールドモデル,SocialForceModel(SFM)などが広く用いられている\cite{helbing_sfm}\cite{sfm_ntt}.
\tensaku{\section{ネットワークモデル}}
ネットワークモデルは,〇〇である.
ネットワークモデルは,災害時における都市部の避難シミュレーションなどの
大域的な解析を高速に解析が可能であるが,群集事故の防止に必要な滞留などの
解析ができないことが報告されている.
このため,建物内などの滞留の解析には,フロアフィールドモデルやSFMが
よく用いられている.(参考文献)
\tensaku{\section{フロアフィールドモデル}}
フロアフィールドモデルは,~~の手法である\cite{floa_field1}\cite{floa_field2}.
群集雪崩などの解析にも用いられる\cite{floa_field3}.



フロアフィールドモデルを用いた避難シミュレーションは,一つの格子が
人の大きさに制約されるため,出口付近で見られる滞留(アーチ現象)の
解析精度が低いことが報告されいている.
高い解析精度が必要な場合は,解析領域を二次元の連続座標として解析する
SFMがよく用いられている(参考文献).
\tensaku{\section{ソーシャルフォースモデル}}
SFMは,社会心理学的な要素と物理学的な要素で成り立つ運動方程式をエージェントごとに計算する
ことで,人流の動きを再現する手法である\cite{helbing_sfm}.
SFMの運動方程式は,目的地に向かう力,周囲のエージェントを避ける力,障害物を避ける力の
合力を算出し,エージェントの速度や進行方向を計算する.
SFMの運動方程式の計算は,エージェントや障害物の数の増加に応じて解析時間が
膨大になることから高速化が求められている.


SFMの避難シミュレーションの文献\cite{sfm_hinan1}\cite{sfm_hinan2}\cite{sfm_hinan3}.

人流シミュレーションのなかでも,歩行者の動きの再現には,視野やグループ特性などの
パラメータを追加できるSocial Force Model(SFM)が広く用いられている
\cite{helbing_sfm},\cite{sfm_ntt},\cite{sfm_para1},\cite{intro_gunshu}
.

視野パラメータを使っている文献\cite{siya_ex2}\cite{siya_ex3}\cite{siya_ex4}\cite{siya_ex5}\cite{siya_ex6}\cite{siya_ex7}


%\cite{siya_ex5},\cite{siya_ex4},\cite{siya_ex6}.

リンクリスト法\cite{cell_book1}\cite{cell_book}\cite{cellrenketu}
ハッシュ法\cite{hash}

視野片寄の文献\cite{katayose}


人流シミュレーションのなかでも歩行者の再現には,ネットワークモデルや
フロアフィールドモデル,ソーシャルフォースモデルなどが解析用途に応じて用いられる.
\tensaku{\section{ネットワークモデル}}
ネットワークモデルは,解析領域をネットワーク構造にモデル化した空間内でエージェント
の動きを再現する手法である.
ネットワークモデルのエージェントは,○○のように動くことで解析する.
このため,ネットワークモデルを用いた解析は,計算負荷が低いため,
都市部における地震や津波などの避難時の人々の流量などを解析することに
用いられることが多い(参考文献).
一方で,ネットワークモデルは,空間をネットワーク構造で再現しているため,
人々が滞留する様子や場所を明確に再現することができない.
人々の滞留を再現するためには,フロアフィールドモデルやSFMを
用いて解析することが一般的である(参考文献).



\tensaku{\section{フロアフィールドモデル}}
フロアフィールドモデルは,エージェントの大きさで格子状に分割した解析領域を
用いて解析する手法である.
フロアフィールドモデルのエージェントは,あらかじめ設定した確率を用いてエージェントを
動かすことで,目的地まで進む.
フロアフィールドモデルを用いた人流シミュレーションは,他のモデルと比べて
解析負荷が低いことが知られている(参考文献).
一方で,本手法は,エージェントの動きが格子の大きさに制約されるため,
出口付近などで発生する滞留(アーチ現象)の再現度が低いことが知られている.
このため,高い精度が求められる解析では,二次元連続空間状で解析できるSFMを
用いることが多い(参考文献).
%要検討(自分の手法をここに書くかどうか)
\tensaku{\section{提案手法}}
壁多い
⇒障害物の計算が多い

壁の特徴
a

そこで!!
~~する
建物内の避難時におけるシミュレーションは,壁や机といった障害物が多く,
障害物を避ける力の計算回数が多い傾向がある.
障害物を避ける力は,エージェントの座標と障害物の座標を用いて計算する.
壁や机などの障害物は,解析中に座標が変化しない特徴がある.
そこで,本論文は,解析領域を格子に分割し,格子領域ごとにあらかじめ
進行方向を計算し,メモリ領域に保存することで,解析中の進行方向の
計算回数を削減する.


%過去
商業施設やイベント会場などの人が多く集まる場所では,
災害時の逃げ遅れの観点から
人の滞留の対策が重要であり\cite{taisaku1},\cite{taisaku2},
人の滞留や避難時間の予測に人流シミュレーションが用いられている
\cite{sim_jirei1},\cite{sim_jirei2},\cite{sim_jirei3},\cite{sim_jirei8}.
人流シミュレーションは,コンピュータ上で人を運動方程式に基づくエージェントとして
解析する手法である.
人流シミュレーションのなかでも,歩行者の動きの再現には,視野やグループ特性などの
パラメータを追加できるSocial Force Model(SFM)が広く用いられている
\cite{helbing_sfm},\cite{sfm_ntt},\cite{sfm_para1},\cite{intro_gunshu}.
%\cite{siya_ex5},\cite{siya_ex4},\cite{siya_ex6}.

%\subsection{ソーシャルフォースモデル}
SFMは,社会心理学的な要素と物理学的な要素で成り立つ運動方程式をエージェントごとに計算する
ことで,人流の動きを再現する手法である.
SFMの運動方程式は,目的地に向かう力,周囲のエージェントを避ける力,障害物を避ける力の
合力を算出し,エージェントの速度や進行方向を計算する.
SFMの運動方程式の計算は,時間ステップごとに全てのエージェントに対して計算するため,
エージェント数の増加するほど,解析時間が膨大になることから高速化が求められている.

%\subsection{ソーシャルフォースモデルの高速化技法}
SFMでは,解析時間の高速化をするために,
モデルの1次元化や
エージェント間距離の計算回数の削減が行われている.
%
%\subsubsection{モデルの単純化}
SFMは,エージェントの動きを1次元に簡略化することで,
計算負荷を削減できる
\cite{1ji_sfm1},\cite{1ji_sfm2}.
SFMの1次元化は,避難人数や避難時間などの解析に対して許容できる
範囲の誤差で高速に解析ができるが,滞留の様子や人の密度などの解析ができないことが
報告されている\cite{1ji_sfm1}.
%
%\subsubsection{エージェント間距離の計算回数の削減}
エージェントを避ける力の計算には,エージェント間の距離が必要である.
エージェント間距離の計算回数の削減には,影響半径の設定や,
セル分割法が広く用いられる
\cite{seru_sfm1},\cite{seru_sfm2},\cite{katayose}.
影響半径の設定は,周囲のエージェントを避ける力や障害物を避ける力の
影響力が遠くなるほど0に近づく特性を利用し,影響半径外から受ける力を0に
近似することで,エージェント間距離の計算回数を減らす手法である.
セル分割法は,解析領域を格子状のセルに分割し,
周囲のエージェントに対する影響範囲内外の判定をセル単位で実行する方法である.
影響範囲内外の判定には,エージェント間距離の計算が必要となるため,
セル単位で判定することで,エージェント間距離の計算回数を削減する.
%\subsection{提案手法}
避難時を再現する人流シミュレーションは,机や壁などの障害物が多いため,
障害物を避ける力の計算回数が多い傾向がある.
机や壁などの固定された物である障害物や目的地は,解析中に座標が変化しないという特徴があり,
目的地まで向かう力を計算するために必要なエージェントから目的地までのベクトルは,
エージェントの座標に応じて決定するという特徴がある.
そこで,本論文では,解析前に目的地までの方向と障害物を避ける力の計算をあらかじめ計算し,
メモリに格納することで,
解析中の障害物を避ける力の計算と目的地までのベクトルの計算回数を削減する手法を提案する.
提案手法は,障害物が固定である特徴と目的地までのベクトルがエージェントの座標に応じて決まる特徴に
着目し,解析領域を格子状に分割した領域ごとに進行方向をあらかじめ計算する.
%}}}
\fi 



%***** END ************************************************

%------------------------------------------------------------------
% 注意点
%------------------------------------------------------------------
% 「はじめに」は,文章の導入部です.
% 論理構成を意識し,3年生にも分かるような説明を心がけましょう.
% 
% 研究背景について述べる必要があるため,サーベイの内容を多く書くことになります.
% 参考文献を多く挙げるようにしましょう.
% 
%
% ※当研究室では2ページ以上書かかないと前川先生にOKがもらえません. 
% 
% 
% 
% 一般的には以下のような構成になると思います.
% [1段落目]
% 背景&需要について書きます.
% この研究によって誰が喜ぶのかが分かるように書きましょう.
% [2段落目]
% 従来手法とその問題点について書きます.
% 1段落目の需要があるのに,誰も研究していないなんてことはありえません.
% [3段落目\UTF{FF5E}]
% 従来手法に対する改良手法について書きます.
% 従来手法の問題点が明らかになっているのに,誰も解決しようとしていないなんてありえません.
% 場合によっては手法ごとに段落を分けて書くこともあると思います.
% 研究に直接的に関係のないサーベイ内容も,ここにならがんがん書きましょう.
% [4段落目]
% 従来の改善方法で十分でしょうか?
% 不十分ならどのような手法が必要でしょうか?提案してください.
% [5段落目]
% 提案手法の手順や特徴を述べます.
% [6段落目]
% 「以降の章では\UTF{FF5E}」の文を書きます.
% 箇条書きになりやすいので,注意しましょう.
% 
%------------------------------------------------------------------
