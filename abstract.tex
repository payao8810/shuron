%%%%%%%%%%%%%%%%%%%%%%%%%%%%%%%%%%%%%%%%%%%%%%%%%%%%%%%%%%%%%%%%%%
% 前研スタイルファイル 設定部分
%
%    27行目までの項目を用いて表紙を生成しています.
%    学科のabstractスタイルが変更になった場合は,
%    27行目までの項目がすべてそろうようにスタイルファイルに
%    付け足してください
%
%%%%%%%%%%%%%%%%%%%%%%%%%%%%%%%%%%%%%%%%%%%%%%%%%%%%%%%%%%%%%%%%%%

%改行位置はそれぞれ自分で調節する
\表紙題目{格子分割を用いた進行方向計算の削減による\\人流シミュレーションの高速化}       

%名前の間は半角スペース
\和文氏名{片寄 颯人}
\英文氏名{KATAYOSE Hayato}

\学生番号{2281011}

\令和年度{5}
\西暦年度{2023}

\提出日{2023年12月24日} % 表紙だけじゃなくて謝辞でも使っているので注意

\和文題目{格子分割を用いた進行方向計算の削減による\\人流シミュレーションの高速化}


%%%%%[ここから下は修士のみ記入]%%%%%%%%%%%%%%%%%%%%%%%%%
%
%   アブストラクト用の設定
%       表紙と改行位置が異なる場合があるので注意!

%\修論false        %この行を消去(コメントアウトでもOK)すること

\英文題目{Speed-up of Pedestrian Simulation by Reduction of \\Direction Calculations using Grid Division}

\和文キーワード{マルチエージェントシミュレーション,人流シミュレーション}
\英文キーワード{Multi Agent Simulation,Pedestrian Simulation,Social Force Model}
\和文論文要旨={
本論文は,SFM(Social Force Model)を用いた人流シミュレーションを高速化するために, エージェントの進行方向計算を削減する手法を提案する.
SFMは,時間ステップごとに各エージェントの運動方程式を解くことで,人々の流れを解析する手法である.
SFMの運動方程式は,目的地に向かう力,周囲のエージェントを避ける力,障害物を避ける力の合力を
用いてエージェントの移動を決定する.
目的地に向かう力や障害物を避ける力は,エージェントの座標に応じて
決定する特徴がある.
そこで,本論文では,解析がを格子状に分割し,格子領域ごとに進行方向をあらかじめ計算することで,
解析中の進行方向の計算回数を削減する.
評価の結果,提案手法は,従来のセル分割法に対して解析時間が最大3.24倍高速化することを確認した.
}% 600字程度


\英文論文要旨={

Write summary here.

}% about 200 words
