%%%%%%%%%%%%%%%%%%%%%%%%%%%%%%%%%%%%%%%%%%%%%%%%%%%%%%%%%%%%%%%%%%
% 前研スタイルファイル 設定部分
%
%    27行目までの項目を用いて表紙を生成しています.
%    学科のabstractスタイルが変更になった場合は,
%    27行目までの項目がすべてそろうようにスタイルファイルに
%    付け足してください
%
%%%%%%%%%%%%%%%%%%%%%%%%%%%%%%%%%%%%%%%%%%%%%%%%%%%%%%%%%%%%%%%%%%

%改行位置はそれぞれ自分で調節する
%\表紙題目{格子分割を用いた進行方向計算の削減による\\人流シミュレーションの高速化}
%\表紙題目{進行方向の計算回数削減による\\SFMの高速化に関する研究}
%\表紙題目{進行方向の計算回数削減による\\SFMを用いた人流シミュレーションの高速化}
\表紙題目{
%進行方向の計算回数削減による\\SFMを用いた人流シミュレーション\\の高速化
進行方向の計算回数削減による\\ソーシャルフォースモデルを用いた\\人流シミュレーションの高速化
}

%名前の間は半角スペース
\和文氏名{片寄 颯人}
\英文氏名{KATAYOSE Hayato}

\学生番号{2281011}

\令和年度{5}
\西暦年度{2024}

\提出日{2023年12月25日} % 表紙だけじゃなくて謝辞でも使っているので注意

%\和文題目{格子分割を用いた進行方向計算の削減による\\人流シミュレーションの高速化}

\和文題目{
進行方向の計算回数削減による\\ソーシャルフォースモデルを用いた\\人流シミュレーションの高速化
}


%%%%%[ここから下は修士のみ記入]%%%%%%%%%%%%%%%%%%%%%%%%%
%
%   アブストラクト用の設定
%       表紙と改行位置が異なる場合があるので注意!

%\修論false        %この行を消去(コメントアウトでもOK)すること

\表紙西暦{2024}

\英文題目{
%Speedup of Pedestrian Simulation by Reduction of \\Direction Calculations using Grid Division
Reducing The Number of Traveling Direction Calculations \\to Speed up for Pedestrian Simulation \\with Social Force Model
}

\和文キーワード{マルチエージェントシミュレーション,人流シミュレーション}
\英文キーワード{Multi Agent Simulation,Pedestrian Simulation,Social Force Model}

\和文論文要旨={
本論文は,ソーシャルフォースモデル(SFM)を用いた人流シミュレーションを高速化することを目的とする.
SFMは,各エージェントの運動方程式を解くことで,人々の流れを解析する手法あり,
必要に応じて視野などのパラメータを追加できることから広く利用されている.
SFMの運動方程式は,目的地に向かう力,周囲のエージェントを避ける力,障害物を避ける力の合力を用いて
エージェントの進行方向を決定する.
SFMを用いた人流シミュレーションは,解析人数や壁などの障害物数が増えるほど,
エージェントの進行方向の計算回数が増加し,解析時間が膨大になるため,高速化が求められている.
そこで,本論文では,SFMを用いた人流シミュレーションを高速化するために,
エージェント間距離および
進行方向の計算回数を削減する手法を提案する.
エージェント間距離の計算回数削減は,
エージェントの進行方向と視野角の関係性を用いて,
視野外に存在するエージェントに対する不要な計算を減らす.
進行方向の計算回数削減は,
目的地や障害物の座標が変わらない特徴を利用し,
解析領域を格子状に分割した格子領域ごとに進行方向をあらかじめ計算する.
評価の結果,本提案手法は,従来のセル分割法に対して,
エージェント間距離の計算回数削減で最大1.25倍,
進行方向の計算回数削減で最大1.80倍高速化することを確認した.
}

\if 0
\和文論文要旨={
駅や商業施設などの人が多く集まる場所では,利便性や災害時の逃げ遅れ防止の観点から
混雑や滞留の対策が重量であり,混雑や滞留の対策にソーシャルフォースモデル(SFM)を用いた人流シミュレーション
が広く利用されている.
SFMは,時間ステップごとに各エージェントの運動方程式を解くことで,人々の流れを解析する手法である.
SFMの運動方程式は,目的地に向かう力,周囲のエージェントを避ける力,障害物を避ける力の合力を用いて
エージェントの移動を決定する.
SFMを用いた人流シミュレーションは,解析人数や壁などの障害物数が増えるほど,
解析時間が膨大になるため,高速化が求められている.
そこで,本論文では,SFMを用いた人流シミュレーションを高速化するために,
エージェントの進行方向の計算回数を削減する手法を提案し,その有効性を評価する.
まず,第1章では,本研究における背景および従来手法について述べ,本研究の目的や
位置づけを明らかにする.第2章では,SFMの計算方法や位置づけなどを述べる.
第3章では,SFMを用いた人流シミュレーションで一般的に用いられる高速化手法に
ついて述べる.
第4章では,エージェント間距離の計算回数を削減することで,
視野を用いたSFMを高速化する手法を提案する.
第5章では,格子分割を用いた進行方向の計算回数を削減することで,
SFMを用いた人流シミュレーションを高速化する手法を提案する.
最後に第6章では,評価の結果から論文全体を総括する.
(648文字)
}

\fi


\if 0
\和文論文要旨={
本論文は,SFM(Social Force Model)を用いた人流シミュレーションを高速化するために,
エージェントの進行方向の計算回数を削減する手法を提案する.
SFMは,時間ステップごとに各エージェントの運動方程式を解くことで,人々の流れを解析する手法である.
SFMの運動方程式は,目的地に向かう力,周囲のエージェントを避ける力,障害物を避ける力の合力を用いて
エージェントの移動を決定する.
SFMを用いた人流シミュレーションは,解析人数や壁などの障害物数が増えるほど,
解析時間が膨大になるため,高速化が求められている.
周囲のエージェントや障害物を避ける力は,影響範囲内のエージェントや障害物から距離に応じた大きさの力を受けるため,
距離計算を用いた影響範囲内外の判定が必要である.
人流シミュレーションのなかでも避難時の解析は,机や壁などの障害物が多い傾向がある.
机や壁などの固定されている障害物や目的地は,解析中に座標が変化しない特徴があり,
目的地までの向かうベクトルは,エージェントの座標に応じて決定するという特徴がある.
そこで,本論文では,解析領域を格子状に分割し,格子領域ごとに進行方向をあらかじめ計算することで,
解析中の進行方向の計算回数を削減する.
評価の結果,提案手法は,従来のセル分割法に対して,許容できる誤差の範囲で
解析時間が最大〇〇倍高速することを確認した.
}
\fi

%\英文論文要旨={
%hi, there.
%}

%\和文論文要旨={
%本論文は,sfm(social force model)を用いた人流シミュレーションを高速化するために,
%エージェントの進行方向の計算回数を削減する手法を提案する.
%sfmは,時間ステップごとに各エージェントの運動方程式を解くことで,人々の流れを解析する手法である.
%sfmの運動方程式は,目的地に向かう力,周囲のエージェントを避ける力,障害物を避ける力の合力を
%用いてエージェントの移動を決定する.
%sfmを用いた人流シミュレーションは,解析人数や壁などの障害物数が増えるほど,
%解析時間が膨大になるため,高速化が求められている.
%人流シミュレーションのなかでも避難時の解析は,机や壁などの障害物が多い
%傾向がある.
%机や壁などの固定されている障害物や目的地は,解析中に座標が変化しないという特徴があり,
%目的地までに向かう力を計算するために必要なエージェントから目的地までの
%ベクトルは,エージェントの座標に応じで決定するという特徴がある.
%そこで,本論文では,解析領域を格子状に分割し,格子領域ごとに進行方向をあらかじめ計算することで,
%解析中の進行方向の計算回数を削減する.
%評価の結果,提案手法は,従来のセル分割法に対して,許容できる誤差の範囲で
%解析時間が最大〇〇倍高速することを確認した.
%(508文字)
%}% 600字程度
%

\英文論文要旨={
The purpose of this paper is to speed up pedestrian simulation with the Social Force Model(SFM).The SFM is a method to simulate human behavior based on motion equations. The motion equation of the SFM uses the resultant forces of the force to go to the destination, the force from surrounding agents, and the force avoiding the obstructions to determine the agent’s movement. Pedestrian simulation with SFM is required to speed up because analysis time increase as the number of agents and obstacles. Therefore, This paper proposes a speedup method by reducing the number of distance calculations between agents and a speedup method by reducing the number of traveling direction calculations. The proposed method of reducing the number of distance calculations between agents reduces distance calculations between agents using the moving direction of the agent and the field of view. As a result of the evaluation, the proposed method gives us about 1.25 times speedup. The proposed method reduces the number of calculations of the agent’s traveling direction during analysis by dividing the analysis region into a grid and calculating the direction of travel for each grid region in advance. As a result of the evaluation, the proposed method gives us about 1.80 times speedup within an acceptable error range.
(209 words)
}

\if 0
\英文論文要旨={
This paper proposes a speedup method by reducing the number of 
calculations of the agent's traveling direction for the Social Force Model(SFM).
The SFM is a method to simulate human behavior based on motion equations.
%SFMの運動方程式の構成
The motion equation of the SFM uses the resultant forces of the force to go to the destination,
the force from surrounding agents, and the force avoiding the obstructions to determine the agent's
movement.
%高速化の理由
The SFM is required to speed up because analysis time increases as the number of agents and obstacles.
%SFMの特徴を述べる
fixed obstacles and destinations have characteristics that the coordinates do not change under analysis.
%そこで,提案~を提案する
Therefore, the proposed method reduces the number of calculations of the agent's traveling direction 
during analysis by dividing the analysis region into a grid and calculating the direction of travel 
for each grid region in advance.
%評価の結果
As a result of the evaluation, 
the proposed method gives us about [高速化率] times speedup within an acceptable error range.
}% about 200 words
\fi

\if 0
memo

エージェントの進行方向
the traveling direction
エージェントの進行方向の計算
calculations of the agent's traveling direction

目的地へ行く力
power to go to destination
the force towards destination
the force to go to the destination

周囲のエージェントから受ける力
the force from surrounding agents

障害物を避ける力
the force avoiding the obstructions

\fi
