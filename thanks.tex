% vim: set tabstop=4 :
%******************************************************************
\chapter*{謝辞}
\label{sec:thanks}
%******************************************************************

本研究の機会及び素晴らしい実験環境を与えて下さり,
貴重な時間を割いて研究の方向性を御指導頂きました前川仁孝教授に
深く感謝致します.
本研究を進めるにあたり,
日頃から惜しみなく御指導して頂きました富永浩文氏に
心から感謝致します.
研究の方向性をはじめ研究の細部に至るまで数々の有意義な御意見,
御助言を賜わりました中村あすか氏に
感謝致します.
特に,本研究のきっかけを与えて下さり,研究の進め方から文章の
書き方まで丁寧に御指導下さった
北川翔一先輩,塙翔登先輩には
この場を借りて心から深く感謝致します.
貴重な御意見,様々な御提案を頂いた前川研究室の皆様に御礼申し上げます.
最後に,私をここまで育てて下さった家族に深く感謝します.


\thanksend
%************************* END ************************************


%------------------------------------------------------------------
% 謝辞例文集(これが礼儀的にどうなのかは謎.自己責任で使用すること)
%               ※指導教員の名前は必ず書くのが礼儀です.
%------------------------------------------------------------------
%
% 本研究の機会及び素晴らしい実験環境を与えて下さり,
% 貴重な時間を割いて研究の方向性を御指導頂きました○○ ○○教授に
% 深く感謝致します.
%
% 本研究を進めるにあたり,
% 日頃から惜しみなく御指導して頂きました○○ ○○氏に
% 心から感謝致します.
%
% 研究の方向性をはじめ研究の細部に至るまで数々の有意義な御意見,
% 御助言を賜わりました○○ ○○氏に
% 感謝致します.
%
% 特に,本研究のきっかけを与えて下さり,研究の進め方から文章の
% 書き方まで丁寧に御指導下さった
% ○○ ○○氏には
% この場を借りて心から深く感謝致します.
%
% 貴重な御意見,様々な御提案を頂いた××ゼミの皆様に御礼申し上げます.
%
% 最後に,私をここまで育てて下さった家族に深く感謝します.
%
%------------------------------------------------------------------
