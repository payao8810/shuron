% vim: set tabstop=4 foldmethod=marker foldlevel=0 :
%**********************************************************
%\chapter{一般的にはサーベイ内容とかを書く}
\chapter{ソーシャルフォースモデル}
\label{sec:survey}
%**********************************************************

スタイルファイルの使い方について少し述べます.
普通に利用する分には,
拡張子が.texファイルのみを編集するだけで事が足りるように設計しました.
ただし,拡張子が.styファイルの書き換えは制限しません.自由に改変してください.


%----------------------------------------------------------
\section{abstract.tex}
%----------------------------------------------------------
abstract.texを書き換えると表紙およびアブストラクトを生成します.
abstract.tex内のコメントにしたがって書き換えを行ってください.
卒論にはアブストラクトが不要です.
修士のみアブストラクトを作成してください.

また,アブストラクトの設定はshuronABS.styに書いてあります.
表紙の設定はpenguin.styに書いてあります.
困ったときはこれらのファイルを変更してください.

%----------------------------------------------------------
\section{簡単コマンド}
%----------------------------------------------------------

penguin.styの294行目以降には, ショートカットコマンドを記述しました.
気が向いたら使ってやってください.
あくまでショートカットコマンドなので,
penguin.styのコマンドを使わなくても同じ機能を実現することができます.

\begin{itemize}
\item $\setminus$owata
\item $\setminus$ol\{ 数式 \}
\item $\setminus$fig\{ タイトル\}\{ ファイル名\}\{ 図の横幅[cm]\}
\item $\setminus$doublefig\{ タイトル1\}\{ ファイル名1\}\{ 図の横幅1[cm]\}\{ 図と図の間隔[cm]\}\{ title\_2\}\{ file\_name2\}\{ size\_2[cm]\}
\item $\setminus$figref\{ fig:ラベル\}
\item $\setminus$tabref\{ tb:ラベル\}
\end{itemize}


\figref{fig:ulysses16}に,$\setminus$figコマンドを用いて図を貼る例を示します.
\figref{fig:ulysses16}は,
$\setminus$fig\{適当な図を張ってみた\}\{ulysses16\}\{5\}で貼り付けています.
\figref{fig:ulysses16}では,図の横幅が5cmになるように大きさ指定をしています.


また,\figref{fig:test1}と\figref{fig:test2}は,
$\setminus$doublefigコマンドを用いて図を並べた例です.
これらの図は,$\setminus$doublefig\{横並び(左)\}\{test1\}\{2.5\}\{0.5\}\{横並び(右)\}\{test2\}\{2.5\}
で貼り付けています.
$\setminus$doublefigコマンドは,図のタイトル高さを自動調節する機能を持っていません.
このため,タイトルの高さは手動で調節してください.

\fig{適当な図を張ってみた}{ulysses16}{5}
\doublefig{横並び(左)}{test1}{2.5}{0.5}{横並び(右)}{test2}{2.5}

図を入れる時には,段落と段落の間に入れてください.
決して文の途中に図が入ることがあってはいけません.
もし,図を参照しているページと図のページが離れてしまった場合は,
段落の長さが適切でない可能性があります.
フォーマットを変えるのではなく,本文の構成を見直しましょう.


%----------------------------------------------------------
\section{参考文献}
\label{sec:bib}
%----------------------------------------------------------
参考文献を参照する文の例です\cite{1983_Ibaraki}.
参考文献の書き方には,bibファイルを使う方法と使わない方法の2通りがあります.
好きな方を選択し,makefileとmain.texを書き換えてください.

\subsection{bibファイルを使う場合}
main.texとmakefileの書き換えは必要ありません.
bibfile.bibに参考文献の記述例があります.
ciniiやIEEEなどでは文献のbibtex情報が用意されているので,
そのファイルをコピペして使えるのが強みです.
また,本方式を使うと,人力で参考文献情報をソートする必要が無いのでありがたいです.
ただし,参考文献が1つも参照されていないとエラーが生じる模様です\cite{test1}.

以下にFAQを載せておきます.
\begin{itemize}
        \item   {\bf bibファイルって何?bibtexって何?}\\
        使い方はgoogle先生に聞いてください.
        \item   {\bf 参考文献情報を書き換えてもコンパイル結果に反映されない}\\
        main.bblファイルを消去してから再コンパイルしてください.
        \item   {\bf 参考文献スタイルを変更したい}\\
        参考文献のフォーマットを決めるファイルは,sty/ipsjunsrt.bstです.
        本ファイルは,情報処理学会のスタイルファイルです.
        \item   {\bf 名字が1文字の人の表示がおかしい}\\
        情報処理学会フォーマットの仕様です.論文提出直前にbblファイルを直接編集してください.
        \item   {\bf bibファイルでエラーが出る}\\
        大抵の場合はカンマ忘れが原因です.次点で参照タグ名の重複かな?
\end{itemize}



\subsection{bibファイルを使わない場合}
自力でthebibliographyの中身を書くパターンです.
bibfile.texに記述例があります.
記述した通りに表示されるため,直感的には分かりやすいです.
ただし,人力での作業量が多くなるので,この方式を使う場合は頑張ってください.

本方式を用いる場合は,以下のファイルの書き換えが必要です.

\begin{itemize}
        \item {\bf main.tex\ }  61行目($\setminus$bibliography\{bibfile\})をコメントアウトし,
                                                62行目($\setminus$input\{bibfile\})のコメントアウトをはずしてください
        \item {\bf makefile\ }  $\#$記号でコメントアウトしてください
        \item {\bf bibfile.tex\ }       ここに参考文献を書いてください.
                                                参考文献は,本文中での参照順番に手動で並び替えが必要です.
\end{itemize}



%***** END ************************************************
